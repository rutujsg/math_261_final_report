\documentclass[12pt,1in]{article}
\usepackage{amsmath, amssymb}
\usepackage{physics}
\usepackage[margin=1in]{geometry}
\usepackage{cite}
\usepackage{booktabs}
\usepackage{graphicx}

\newenvironment{Example}[2][Example]{\begin{trivlist}
		\item[\hskip \labelsep {\bfseries #1}\hskip \labelsep {\bfseries #2.}]}{\end{trivlist}}
\usepackage{hyperref}

\title{Analyzing Non-linear Ordinary Differential Equations\\ {\small MATH 26600}}
\author{Rutuj Gavankar}
\date{}
\begin{document}

\maketitle

\section{Introduction}

\subsection{Linearizing a system of ODE's}


Let 
\begin{align}
\derivative{x}{t} &= F(x, y) \\
\derivative{y}{t} &= G(x , y)
\end{align}
be a system of first-order differential equations. The steady-state solutions of this system are the solutions for which  $x(t)$ and $y(t)$ are invariant. That is to say,
\begin{align}
    F(x,y) &= 0 \\
    G(x,y) &= 0
\end{align}
We can analyzing this system around these equilibrium points by making linear approximations of the function around these points. Using first order Taylor series expansion for $F$ and $G$ we get
\begin{align}
    F(x,y) &\approx F(x_0, y_0) + F_x(x_0, y_0)(x - x_0) + F_y(x_0, y_0)(y - y_0) \\
    G(x,y) &\approx G(x_0, y_0) + G_x(x_0, y_0)(x - x_0) + G_y(x_0, y_0)(y - y_0)
\end{align}
Where $x_0$ and $y_0$ are the equilibrium points for the system. The system can be re-written in matrix-vector notation as 
\begin{align}
    \begin{bmatrix}
    \derivative{x}{t} \\
    \derivative{y}{t}
    \end{bmatrix} &= 
    \begin{bmatrix}
    F_x(x_0, y_0) & F_y(x_0, y_0)\\
    G_x(x_0, y_0) & G_y(x_0, y_0)
    \end{bmatrix}
    \begin{bmatrix}
    x - x_0 \\
    y - y_0
    \end{bmatrix}
\end{align}Now, let $x - x_0$ be $u$ and $y - y_0$ be $v$, and $\Vec{v}$ be the vector $\left<u ,v\right>$
\begin{align}
    \therefore \derivative{\Vec{v}}{t} &= J\Vec{v} \label{eq:vec_mat}
\end{align}
Where $J$ is the Jacobian matrix, $$J = \begin{bmatrix}
    F_x(x_0, y_0) & F_y(x_0, y_0)\\
    G_x(x_0, y_0) & G_y(x_0, y_0)
    \end{bmatrix}$$
Now, Eq. \ref{eq:vec_mat} is an eigenvalue problem. The local stability and the behaviour of the system around the equilibrium points can be inferred from the eigenvalues of $J$. 


\begin{Example}{1} \cite[p.~488]{diff_eq}
	Consider the system for $(x,y \geq 0)$
	\begin{align*}
    \derivative{x}{t} &= x(10 - x - y) \\
    \derivative{y}{t} &= y(30 - 2x - y)
\end{align*}
The Jacobian of the system is 
\begin{align*}
    J = \begin{bmatrix}
    10 - 2x - y & -x \\
    -2y & 30 - 2x - 2y
    \end{bmatrix}
\end{align*}
The system has equilibrium points at $(0,0)$, $(10,0)$, $(0,30)$. Analyzing the system at $(0,0)$,
\begin{align*}
    J|_{(0,0)} &= \begin{bmatrix}
    10 & 0 \\
    0 & 30
    \end{bmatrix}
\end{align*}
Since $J$ is a diagonal matrix, the eigenvalues of $J$ are the elements along its diagonal. That is, $\lambda_{1,2} = \{ 10, 30 \}$. Since both the eigenvalues are real and positive, the point $(0,0)$ is a nodal-source.
Similarly, analyzing the system at $(10,0)$,
\begin{align*}
    J|_{(10,0)} &= 
    \begin{bmatrix}
    -10 & -10\\
    0 & 10 
    \end{bmatrix}
\end{align*}
The eigenvalues of $J$ are $\lambda_{1,2} = \{-10, 10\}$. Since both the eigenvalues are real and nonzero, and $\lambda_1 < 0$, $\lambda_2 > 0$, the point $(10,0)$ is a saddle point. 

\end{Example}

\section{Hamiltonian Systems}
\section{Dissipative Systems}

\bibliography{citations}
\bibliographystyle{plain}

\end{document}
